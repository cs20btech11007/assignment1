\documentclass[journal,14pt,twocolumn]{IEEEtran}

\usepackage{setspace}
\usepackage{gensymb}
\singlespacing
\usepackage[cmex10]{amsmath}

\usepackage{hyperref}

\usepackage{amsthm}

\usepackage{mathrsfs}
\usepackage{txfonts}
\usepackage{stfloats}
\usepackage{bm}
\usepackage{cite}
\usepackage{cases}
\usepackage{subfig}

\usepackage{longtable}
\usepackage{multirow}

\usepackage{enumitem}
\usepackage{mathtools}
\usepackage{steinmetz}
\usepackage{tikz}
\usepackage{circuitikz}
\usepackage{verbatim}
\usepackage{tfrupee}
\usepackage[breaklinks=true]{hyperref}
\usepackage{graphicx}
\usepackage{tkz-euclide}

\usetikzlibrary{calc,math}
\usepackage{listings}
    \usepackage{color}                                            %%
    \usepackage{array}                                            %%
    \usepackage{longtable}                                        %%
    \usepackage{calc}                                             %%
    \usepackage{multirow}                                         %%
    \usepackage{hhline}                                           %%
    \usepackage{ifthen}                                           %%
    \usepackage{lscape}     
\usepackage{multicol}
\usepackage{chngcntr}

\DeclareMathOperator*{\Res}{Res}

\renewcommand\thesection{\arabic{section}}
\renewcommand\thesubsection{\thesection.\arabic{subsection}}
\renewcommand\thesubsubsection{\thesubsection.\arabic{subsubsection}}

\renewcommand\thesectiondis{\arabic{section}}
\renewcommand\thesubsectiondis{\thesectiondis.\arabic{subsection}}
\renewcommand\thesubsubsectiondis{\thesubsectiondis.\arabic{subsubsection}}


\hyphenation{op-tical net-works semi-conduc-tor}
\def\inputGnumericTable{}                                 %%

\lstset{
%language=C,
frame=single, 
breaklines=true,
columns=fullflexible
}
\begin{document}

\newcommand{\BEQA}{\begin{eqnarray}}
\newcommand{\EEQA}{\end{eqnarray}}
\newcommand{\define}{\stackrel{\triangle}{=}}
\bibliographystyle{IEEEtran}
\raggedbottom
\setlength{\parindent}{0pt}
\providecommand{\mbf}{\mathbf}
\providecommand{\pr}[1]{\ensuremath{\Pr\left(#1\right)}}
\providecommand{\qfunc}[1]{\ensuremath{Q\left(#1\right)}}
\providecommand{\sbrak}[1]{\ensuremath{{}\left[#1\right]}}
\providecommand{\lsbrak}[1]{\ensuremath{{}\left[#1\right.}}
\providecommand{\rsbrak}[1]{\ensuremath{{}\left.#1\right]}}
\providecommand{\brak}[1]{\ensuremath{\left(#1\right)}}
\providecommand{\lbrak}[1]{\ensuremath{\left(#1\right.}}
\providecommand{\rbrak}[1]{\ensuremath{\left.#1\right)}}
\providecommand{\cbrak}[1]{\ensuremath{\left\{#1\right\}}}
\providecommand{\lcbrak}[1]{\ensuremath{\left\{#1\right.}}
\providecommand{\rcbrak}[1]{\ensuremath{\left.#1\right\}}}
\theoremstyle{remark}
\newtheorem{rem}{Remark}
\newcommand{\sgn}{\mathop{\mathrm{sgn}}}
\providecommand{\abs}[1]{\vert#1\vert}
\providecommand{\res}[1]{\Res\displaylimits_{#1}} 
\providecommand{\norm}[1]{\lVert#1\rVert}
%\providecommand{\norm}[1]{\lVert#1\rVert}
\providecommand{\mtx}[1]{\mathbf{#1}}
\providecommand{\mean}[1]{E[ #1 ]}
\providecommand{\fourier}{\overset{\mathcal{F}}{ \rightleftharpoons}}
%\providecommand{\hilbert}{\overset{\mathcal{H}}{ \rightleftharpoons}}
\providecommand{\system}{\overset{\mathcal{H}}{ \longleftrightarrow}}
	%\newcommand{\solution}[2]{\textbf{Solution:}{#1}}
\newcommand{\solution}{\noindent \textbf{Solution: }}
\newcommand{\cosec}{\,\text{cosec}\,}
\providecommand{\dec}[2]{\ensuremath{\overset{#1}{\underset{#2}{\gtrless}}}}
\newcommand{\myvec}[1]{\ensuremath{\begin{pmatrix}#1\end{pmatrix}}}
\newcommand{\mydet}[1]{\ensuremath{\begin{vmatrix}#1\end{vmatrix}}}
\numberwithin{equation}{subsection}
\makeatletter
\@addtoreset{figure}{problem}
\makeatother
\let\StandardTheFigure\thefigure
\let\vec\mathbf
\renewcommand{\thefigure}{\theproblem}
\def\putbox#1#2#3{\makebox[0in][l]{\makebox[#1][l]{}\raisebox{\baselineskip}[0in][0in]{\raisebox{#2}[0in][0in]{#3}}}}
     \def\rightbox#1{\makebox[0in][r]{#1}}
     \def\centbox#1{\makebox[0in]{#1}}
     \def\topbox#1{\raisebox{-\baselineskip}[0in][0in]{#1}}
     \def\midbox#1{\raisebox{-0.5\baselineskip}[0in][0in]{#1}}
\vspace{3cm}
\title{AI1103-Assignment 1}
\author{Name: badavathu revanth, Roll Number: CS20BTECH11007}
\maketitle
\newpage
\bigskip
\renewcommand{\thefigure}{\theenumi}
\renewcommand{\thetable}{\theenumi}
Download all python codes from 
\begin{lstlisting}
https://github.com/cs20btech11007/assignment1/blob/main/assignment1/code/assignment1.py



\end{lstlisting}
%
and latex-tikz codes from 
%
\begin{lstlisting}
https://github.com/cs20btech11007/assignment1/blob/main/assignment1/main.tex

\end{lstlisting}
\begin{center}
\section*{Problem 1.12}
\end{center}
A coin is biased so that the head is 3 times
as likely to occur as tail. If the coin is tossed
twice, find the probability distribution of
number of tails.\\

\subsection*{sol.}
 In the question given that,coin is biased and head is 3 times as likely to occur as tail.
The coin has tossed twice 
let X be random varaible X$\in\{0,1,2 \}$ denotes outcomes  in a experiment showing number of tails.\\
$P(X)_h$   denotes the out come is head.\\
$P(X)_t$    denotes the out come is tail .\\
Given that,
\begin{align}
P(X)_h=3P(X)_t
 \end{align}

and we know that,
\begin{equation}
 P(X)_{h} +P(X)_{t}=1\\
\end{equation}\\

substitute in eqn (1)$$P(X)_h=3P(X)_i $$ 
$$P(X)_t+3P(X)_t=1$$
$$4P(X)_t=1$$
$$ P(X)_t=\frac{1}{4}$$
and $$P(X)_h=\frac{3}{4}$$.\\


  using  binomial distribution and now finding probability distribution of number of tails in the events.\\
  \begin{align}\label{equation-1}
   \pr{X=k}_t =
  \begin{cases}
    {^n C_k}p^{k}\brak{1-p}^{n-k} \\& 0 \leq k \leq n\\
      0 & otherwise
  \end{cases}
 \end{align}
\begin{equation}
\pr{X=k}_t={^n C_k}( \frac{1}{4})^{n} (1-\frac{1}{4}) ^{n-k}
\end{equation}\\

$Pr(X=0)_t $ denotes the " 0 " number of times the coins showed tail in 2 tosses.\\
\begin{align}
 \pr{X=0}_t={^2 C_0}\brak{ \frac{1}{4}^{0}} \brak{1-\frac{1}{4}} ^{2}=\left(\frac{3.3}{4.4}\right)=\frac{9}{16}\\
 \implies \pr{X=0}_t=0.5625 \nonumber
 \end{align}
\\$Pr(X=1)_t $ denotes the " 1 " number of times the coins showed tail in 2 tosses.\\
\begin{align}
\pr{X=1}_t={{^2 C_1}\brak{\frac{1}{4}}^{1} \brak{1-\frac{1}{4}} ^{1}=(2)\left(\frac{1.3}{4.4}\right)}=\frac{6}{16}\\
\implies \pr{X=1}_t=0.375 \nonumber
 \end{align}
\\$Pr(X=2)_t $ denotes the " 2 " number of times the coins showed tail in 2 tosses.\\
\begin{align}
\pr{X=2}_t={^2 C_2}\brak{ \frac{1}{4}}^{2} \brak{1-\frac{1}{4}}^{0}=\left(\frac{1.1}{4.4}\right)=\frac{1}{16}\\
 \implies \pr{X=2}_t=0.0625 \nonumber
 \end{align}
\begin{table}[h]
\resizebox{\columnwidth}{!}{
    \begin{tabular}{|c|c|c|}
        \hline
        X & \pr{X=k}_t\\
        \hline
        0 & \pr{X=0}_t=$0.5625$ \\
        \hline
        1 & \pr{X=1}_t=$0.375$\\
        \hline
        2 & \pr{X=2}_t=$0.0625$\\
        \hline
        \end{tabular}
}
\\
\caption{Outcome of the Experiment} 

\end{table}

\end{document}

